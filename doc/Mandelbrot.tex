\documentclass[a4paper]{ctexart}

\usepackage{xltxtra}
\usepackage{amsthm}
\usepackage{amsmath}
\usepackage{amssymb}
\usepackage{graphicx}
\usepackage{cite}
\usepackage{tikz}
\usepackage{xcolor}
\usetikzlibrary{positioning, shapes.geometric}


\title{Mandelbrot Set的生成和探索}
\author{蔡聪聪\\信息与计算科学 3180102279}
\date{}
\bibliographystyle{plain}


\begin{document}

\maketitle

\begin{abstract}
 这是一个摘要\par
\textbf{关键字:}这是一个关键字
\end{abstract}

这是一段引言

\section{问题的背景介绍}
\verb|Mandelbrot Set|是一个几何图形,曾被称为“上帝的指纹”.这个点集均出自公式: $z_{n+1}=z_n^2+c$,对于非线性迭代公式 $z_{n+1}=z_n^2+c$,所有使得无限迭代后的结果能保持有限数值的复数 \verb|z|的集合连通的 \verb|c| 构成 \verb|Mandelbrot Set|.\cite{ztr2013hd}它是曼德勃罗教授在二十世纪七十年代发现的.
\section{数学理论}
本文的算法基于以下理论.
\newtheorem{lemma}{定理}
\begin{lemma}
  $c \notin M$当且仅当存在$n \in N^+$,使得$|z_n|>2$.
\end{lemma}

\begin{proof}
分别探讨$|c|>2$与$|c|\leq 2$两种情形,首先证明$|c| \leq 2$时的情况:\par
假设$|z_n|>2$.因为$|c| \leq 2$,故$|z_n|>|c|$.
因为$|z_n|>|c|,|z_n|>2$,故$|z_{n+1}|=|z_n^2+c|\geq|z_n|^2-|c|>|z_n|^2-|z_n|>2|z_n|-|z_n|=|z_n|$.
由以上可知$|z_{n+1}|>|z_n|$.由数学归纳法可知$2<|{z_n}|<|z_{n+1}|<|z_{n+2}|<...,$ 可看出随着迭代次数增加$|z_n|$逐渐递增并发散.即$c\notin M$\par
同理,$|z_{n}|>2,(n=1,2,...)$且$|c|>2$时,$c\notin M$.\par
综合上述可得知不论$|c|$为多少,若$|z_n|>2$,则$c\notin M$.
\end{proof}

\section{算法:计算Mandelbrot Set并生成黑白图像}
\centering
\begin{tikzpicture}[node distance=2cm]
  \node[draw, rounded corners]                        (start)   {定义初始数据:复数域C,最大迭代次数N};
  \node[draw, below=of start]                         (step 1)  {$n=0,z=0$.对每个$c\in C$};
  \node[draw, below=of step 1]                        (step 2)  {$z=z^2+c, n=n+1$};
  \node[draw, diamond, aspect=2, below=of step 2]     (choice)  {$n>N?$};
  \node[draw, diamond, aspect=2,right=30pt of choice] (step x)  {$|z|<2?$};
  \node[draw, rounded corners, below=30pt of choice]  (end1)     {c点为黑色};
  \node[draw, rounded corners, below=30pt of step x]  (end2)     {c点为白色};
  
  \draw[->] (start)  -- (step 1);
  \draw[->] (step 1) -- (step 2);
  \draw[->] (step 2) -- (choice);
  \draw[->] (choice) -- node[left]  {是} (end1);
  \draw[->] (choice) -- node[above] {否} (step x);
  \draw[->] (step x) -- node[right] {是} (step x|-step 2) ->  (step 2);
  \draw[->] (step x) -- node[left]  {否} (end2);
\end{tikzpicture}

\section{数值算例}
图片
\section{结论}

\bibliography{books}
\end{document}
